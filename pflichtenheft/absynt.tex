
\section{Abstrakte Syntax}
\label{sec:AbstrakteSyntax}

\team{Karsten Stahl, Martin Steffen, und alle anderen}

Folgende \emph{erweiterte BNF}-Notation fa�t die \emph{abstrakte Syntax}
als gemeinsame Zwischenrepresentierung zusammen. Abgesehen von einigen
Namenkonventionen (Gro�schreibung) ist die Umsetzung in \Java{} trivial.
Jeder nichtterminale Eintrag wird ein \emph{Klasse.} Alternativen,
gekennzeichnet durch $\bnfbar$, sind Unterklassen der \emph{abstrakten
  Klasse,} deren Unterf�lle sie bilden. Die Eintr�ge der mittleren Spalte
werden als \emph{Felder} der Klassen repr�sentiert. Die Konstruktoren sind,
bis auf die Reihenfolge der Argumente, durch die Felder der Klasse
festgelegt.\footnote{Es gibt Ausnahmen von der letzten Regel, n�mlich f�r
  die (\Snot-)Typen in den Ausdr�cken. Die Typen sind nicht in die
  Konstruktoren mit aufgenommen. Die entsprechenden Felder werden
  nachtr�glich eingetragen.} Die \emph{Listen} der EBNF wurden als
\texttt{java.lang.LinkedList}-Klassen implementiert. Graphische Information
zur Positionierung, die nur f�r den Editor relevant ist, wurde nicht mit in
die EBNF des Pflichtenheftes mit aufgenommen.


\medskip

\newpage
\inputcode{sfc-absynt.txt}
%\lstinputlisting{sfc-absynt.txt}
%\lstinputlisting{#1/#2.java}  \lstinputlisting{#1/#2.java}




%%% Local Variables: 
%%% mode: latex
%%% TeX-master: "main"
%%% End: 
