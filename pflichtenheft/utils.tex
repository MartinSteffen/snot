\section{Hilfsprogramme}
\label{sec:utils}

Verschiedene Programme, die keinem anderen Paket zugeteilt sind und die
mehreren Paketen n�tzen.


\subsection{Pretty-Printer}
\label{sec:prettyprinter}

\team{}

Ein einfacher Pretty-Printer mit tabuliertem ascii-Output, er soll vor
allem zu Diagnosezwecken dienen. Dieser Teil sollte einfach sein. Es ist
wichtig, da� der Pretty-Printer relativ schnell bereitgestellt ist, da er
das Testen und Debuggen der anderen Teile unterst�tzt.


\subsubsection*{Schnittstelle}

Jeder darf (und soll) den Pretty-Printer benutzen, er dient haupts�chlich
zur Diagnose. Die einzige Schnittstelle die z�hlt ist, da� er abstrakte
Syntax ausgeben k�nnen mu�.  Die Schnittstelle ist bereits teilweise
implementiert (zur Verwendung siehe \texttt{utils.PpExample}). Es werden
neben der \texttt{print}-Funktion f�r ganze Programme gleichlautende
Methoden f�r andere syntaktische Konstrukte zur Verf�gung gestellt
(\texttt{public}), damit man auch von au�en Teilprogramme ausdrucken kann.


%%% Local Variables: 
%%% mode: latex
%%% TeX-master: "main"
%%% End: 
