\section{Checks}
\label{sec:checks}

\team{}

Nur syntaktisch korrekte Systeme k�nnen simuliert und als Basis f�r die
Codegenerierung verwendet werden.  Deshalb soll die syntaktische
Korrektheit �berpr�ft werden.

Die Aufgabe beinhaltet die Definition der syntaktischen Korrektheit, d.h.\ 
der Begriff der Korrektheit (was soll alles gecheckt werden) soll
formuliert und als Modul implementiert werden.




\subsubsection*{Schnittstelle}

Mit der Gui. Die Gui stellt dar�ber hinaus sicher, da� die Pakete
Graphplatzierung, Simulation, Model-Checking und Codegenerierung nur gecheckte
Syntax bekommen. Nicht gecheckt wird ``graphische'' Notation (ob Steps
�bereinanderliegen etc.), daf�r ist der Editor aus Abschnitt~\ref{sec:editor}
da.

Die Schnittstelle sei (zumindest) eine Methode \texttt{start\_check} mit
Parameter eines Objektes der abstrakten Syntax.

Was genau gecheckt wird, bleibt zu \emph{diskutieren!}




%%% Local Variables: 
%%% mode: latex
%%% TeX-master: "main"
%%% End: 
