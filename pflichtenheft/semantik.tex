\section{Informelle Semantik}
\label{sec:semantik}


Dieser
Abschnitt beschreibt informell die Bedeutung der \Siff-Programme. Die
Semantik ist nur f�r \emph{gecheckte} SFC's definiert (s.\ 
Abschnitt~\ref{sec:checks}); nicht-gecheckte Programme sind bedeutungslos.
Insbesondere k�nnen der Simulator und der Model-Checker
(Abschnitt~\ref{sec:simulator} und \ref{sec:assert}), die die Semantik
realisieren, von gecheckter Syntax ausgehen.



\subsection{Zust�nde}


Der globale Zustand eines Programmes ist gegeben durch die Variablenbelegungen
und die Menge der aktiven Steps.  





\subsection{Kommunikationsmodell und Schritte}

Die Bedeutung der Konstrukte wie Wertzuweisung, Anfangszustand etc., sollte
unstrittig sein. Interpretationsspielraum gibt es f�r die
Kommunikationskonstrukte und das Wesen der Parallelit�t.

Aufgrund der informellen Diskussion haben wir uns auf folgendes Modell
geeinigt:

\begin{itemize}
\item Interleaving-Parallelit�t
\item zwei-Weg Kommnuikation (kein Broadcast)
\item synchroner Nachrichtenaustausch
\end{itemize}
Das hei�t genauer folgendes: Der Zustand des Programmes �ndert sich in
den \emph{Transitionen}. Es gibt 4 Arten davon
\begin{enumerate}
\item interne Aktionen (``tau'')
\item Zuweisung
\item Input
\item Output
\end{enumerate}
Die ersten beiden Aktionen werden von einem einzigen Proze� \emph{alleine}
ausgef�hrt, d.h., wir haben eine
\emph{Interleaving}-Interpretation.\footnote{Nicht-Interleaving w�re, wenn
  mehrere gleichzeitig Wertzuweisungen und $\tau$-Aktionen machen
  k�nnten.} Die $\tau$-Aktion l��t die Variablenbelegung unver�ndert,
die Verwertzuweisung �ndert sie. Daneben wechselt der Betroffene Prozess
von einem \textit{state} und einen Nachfolgezustand entsprechend der
Transition.


Daneben gibt es zwei Arten von \emph{Kommunikationstransitionen} (input und
output) die immer \emph{komplement�r und gleichzeitig} ausgef�hrt werden
(synchones Modell) und bei der immer genau zwei Partner beteiligt sind
(zwei-Wege Kommunikation).\footnote{Falls es ``grunds�tzlich'' keinen
  Kommunikationspartner gibt, wird anngenommen, da� Kommunikation mit der
  \emph{Umgebung} gemeint ist. Das ist f�r den Simulator von Bedeutung,
  bei dem der Benutzer die Rolle der Umgebung spielen kann. Siehe die
  Diskussion dort.} Eine Kommunikation zwischen zwei Prozessen kann dann
stattfinden, wenn sich beide in einem Zustand befinden, bei dem der eine
eine Eingabe-, der andere eine Ausgabeaktion durchf�hren k�nnen (d.h.
auch, ihre Guards evaluieren zu \emph{true}). Falls da� der Fall ist,
wechseln \emph{beide synchron} in ihren Nachfolge-state, wobei der
Empf�ngerproze� auch noch seine Variablenbelegung durch den empfangenen
Wert �ndert. 




Die Transitionen k�nnen mit einem \textit{Guard} ausgestattet sein, einem
\emph{booleschen Ausdruck.}  Eine Transition kann nur genommen werden,
falls der Guard sich zu \emph{true} evaluiert.




%%% Local Variables: 
%%% mode: latex
%%% TeX-master: "main"
%%% End: 
