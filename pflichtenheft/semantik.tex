\section{Informelle Semantik}
\label{sec:semantik}


Dieser
Abschnitt beschreibt informell die Bedeutung der Sequential Function Charts
(SFC's), f�r die das Tool \Snot{} entwickelt werden soll. Die
Semantik ist nur f�r \emph{gecheckte} SFC's definiert (s.\ 
Abschnitt~\ref{sec:checks}); nicht-gecheckte SFC's sind bedeutungslos.
Insbesondere k�nnen der Simulator und der Model-Checker
(Abschnitt~\ref{sec:simulator} und \ref{sec:assert}), die die Semantik
realisieren, von gecheckter Syntax ausgehen.



\subsection{Sequential Function Charts}

Wir erl�utern die Semantik der SFC's anhand des folgenden Beispiels:


% Macros for SFC pictures

\newcommand{\GUARD}[1][]{%
 \put(-1.5,0){\linethickness{0.2\unitlength}\line(1,0){3}}%
              \put(2.5,0){\makebox(0,0)[l]{#1}}}
\newcommand{\STEP}[1][]{{%
% \put(-2.5,-2.5){\rule{5\unitlength}{5\unitlength}}%
 \put(-2.5,-2.5){\linethickness{0.2\unitlength}%
                 \framebox(5,5){#1}}}}
\newcommand{\STEPinitial}[1][]{{%
% \put(-2.5,-2.5){\rule{5\unitlength}{5\unitlength}}%
 \put(-2.5,-2.5){\linethickness{0.2\unitlength}%
                 \framebox(5,5){#1}}}
 \put(-2.2,-2.2){\linethickness{0.2\unitlength}%
                 \framebox(4.4,4.4){}}}
\newcommand{\LINE}{\linethickness{0.15\unitlength}\line}
\newcommand{\VECTOR}{\thicklines\linethickness{0.15\unitlength}\vector}

% Mark a step as active
\newcommand{\Mark}{\put(-3,-3){\linethickness{0.5\unitlength}%
                   \framebox(6,6){}}}

% Action block
\newcommand{\AB}[3][\put(2.5,1){\line(1,0){2}}]{\linethickness{0.15\unitlength}
  #1
  \put(4.5,-0.5){\rule{12\unitlength}{3\unitlength}}
  \put(4.5,2.5){\line(1,0){12}}\put(4.5,-0.5){\line(1,0){12}}
  \put(4.5,-0.5){\line(0,1){3}}\put(7.0,-0.5){\line(0,1){3}}
  \put(16.5,-0.5){\line(0,1){3}}
  \put(5.75,1){\makebox(0,0){\small\texttt{#2}}}
  \put(11.75,1){\makebox(0,0){\small\textit{#3}}}
}

% Mark an action as active
\newcommand{\MarkA}{\put(7,-0.5){\linethickness{0.4\unitlength}%
                    \framebox(9.5,3){}}}

% Action block with timing constraint
\newcommand{\ABt}[3][]{\linethickness{0.15\unitlength}
  \put(2.5,1){\line(1,0){2}}
  \put(4.5,-1){\rule{14.5\unitlength}{4\unitlength}}
  \put(4.5,3){\line(1,0){14.5}}\put(4.5,-1){\line(1,0){14.5}}
  \put(4.5,-1){\line(0,1){4}}\put(9.5,-1){\line(0,1){4}}
  \put(19,-1){\line(0,1){4}}
  \put(7,2.7){\makebox(0,0)[t]{\small\texttt{#2}}}
  \put(7,-0.8){\makebox(0,0)[b]{\tiny\textrm{#1}}}
  \put(14.25,1){\makebox(0,0){\small\textit{#3}}}
}

% Mark a timed action as active
\newcommand{\MarkAt}{\put(9.5,-1){\linethickness{0.4\unitlength}%
                     \framebox(9.5,4){}}}



\newcommand{\INSTEP}[1]{\parbox{3cm}{\texttt{~~~~%
        \begin{tabular}{|c|l|} \hline
          N & #1 \\ \hline
        \end{tabular}
}}}
\newcommand{\INGUARD}[1]{{\texttt{~~#1}}}


\unitlength1.8mm%
\begin{picture}(44.5,75)(-27,-15)
  \put(  0, 55){\put(  0,-2.5){\LINE(0,-1){6}}
    \put(  0,-5.5){\GUARD{\INGUARD{true}}}
    \put(-12,-8.5){\LINE(1,0){24}}
    \put(-12,-9.5){\LINE(1,0){24}}
    \put(-12,-8.5){\LINE(0,-1){4}}
    \put( 12,-8.5){\LINE(0,-1){4}}
    \put( 2,0){\LINE(1,0){2}}
    \STEPinitial[$s_1$]{\INSTEP{x\,:=\,false}}
    }
  \put(-12, 40){\put(  0,-2.5){\LINE(0,-1){25}}
    \put(  0,-14.5){\GUARD{\INGUARD{x and y}}}
    \put( 2,0){\LINE(1,0){2}}
    \STEP[$s_2$]{\INSTEP{y\,:=\,x}}
    }
  % oberster im rechten zweig
  \put( 12, 40){\put(  0,-2.5){\LINE(0,-1){4}}
    \put( -5,-6.5){\LINE(1,0){10}}
    \put( -5,-6.5){\LINE(0,-1){6}}
    \put( -5,-9.5){\GUARD{\INGUARD{y}}}
    \put(  5,-6.5){\LINE(0,-1){6}}
    \put(  5,-9.5){\GUARD{\INGUARD{not y}}}
    \put( 2,0){\LINE(1,0){2}}
    \STEP[$s_3$]{\INSTEP{x\,:=\,not x}}
    }
  % splitten im rechten zweig
  \put( 7, 25){\put(  0,-2.5){\LINE(0,-1){6}}
    \put(  0,-5.5){\GUARD{\INGUARD{true}}}
    \STEP[$s_5$]{}
    }
  \put( 17, 25){\put(  0,-2.5){\LINE(0,-1){6}}
    \put(  0,-5.5){\GUARD{\INGUARD{true}}}
    \STEP[$s_6$]{}
    }
  % zusammenfuehrung rechts nach splitten
  \put( 7, 19){\put(  0,-2.5){\LINE(1,0){10}}
    \put(  5,-2.5){\LINE(0,-1){4}}
    }
  \put(  12, 10){
    \STEP[$s_7$]{}
    \put(  0,-2.5){\LINE(0,-1){4}}
    }

  % unterer step im linken zweig + parallel-konstrukt schliessen
  \put(-12, 10){\put(  0,-2.5){\LINE(0,-1){4}}
    \put(  0,-5.5){\LINE(1,0){24}}
    \put(  0,-6.5){\LINE(1,0){24}}
    \put( 12,-6.5){\LINE(0,-1){6}}
    \put( 12,-9.5){\GUARD{\INGUARD{true}}}
    \STEP[$s_4$]{}
    }
  % unterster step + nach oben
  \put(  0, -5){\put(  0,-2.5){\LINE(0,-1){6}}
    \put(  0,-5.5){\GUARD{\INGUARD{true}}}
    \put(  0,-8.5){\LINE(-1,0){20}}
    \put(-20,-8.5){\LINE(0,1){68.5}}
    \put(-20,60  ){\VECTOR(1,0){17.5}}
    \STEP[$s_8$]{}
    }
\end{picture}


Die SFC's bestehen aus Knoten, genannt \emph{Steps}, zu denen \emph{Aktionen}
assoziiert sind, sowie aus \emph{Transitionen} zwischen Steps, die mit
booleschen \emph{Guards} versehen sind.  Es sind immer einer oder mehrere der
Steps aktiv; die mit diesen aktiven Steps assoziierten Aktionen werden in
einem Arbeitszyklus ausgef�hrt.  Die Transition von $s_1$ zu $s_2$ und $s_3$
(mit doppelter horizontaler Linie) ist eine \emph{parallele} Verzweigung, wird
diese Transition genommen, so wird $s_1$ deaktiviert und $s_2$ sowie $s_3$
aktiviert.

Der oberste speziell markierte Step ist initial.  Das "`N"' vor den Aktionen
ist ein \emph{Qualifier}, er besagt, dass die Aktion in jedem Arbeitszyklus
ausgef�hrt werden soll, in dem der Step aktiv ist.  Es gibt noch weitere
Qualifier, die wir aber erst einmal vernachl�ssigen.

Der Ablauf eines SFC's (ein \emph{Zyklus}) ist wie folgt:
\begin{itemize}
\item Inputs lesen von der Umgebung
\item Aktionen der aktiven Steps ausf�hren
\item Guards auswerten
\item Transitionen nehmen (wenn m�glich)
\item Outputs schreiben
\end{itemize}
Dieser Zyklus wird immer wieder abgearbeitet.  Die Schritte \emph{Inputs
  lesen} und \emph{Outputs schreiben} sind f�r uns erst einmal irrelevant, da
wir nur abgeschlossene Systeme betrachten, d.h.  Systeme, deren Variablen nur
durch das System selbst ver�ndert werden.

Die Transitionen sind mit einem \emph{Guard} ausgestattet sein, einem
\emph{booleschen Ausdruck.}  Eine Transition kann nur genommen werden,
falls sich der Guard zu \emph{true} evaluiert.

Sind aufgrund einer parallelen Verzweigung \emph{mehrere} Steps aktiv, so
erfolgt die Ausf�hrung der zugeh�rigen Aktionen nichtdeterministisch, d.h. sie
sind in beliebiger Reihenfolge m�glich (\emph{Interleaving-Semantik}).
Folglich gibt es unter Umst�nden eine Vielzahl verschiedener L�ufe eines
SFC's, abh�ngig von diesen Ausf�hrungsreihenfolgen.  Der Simulator soll dies
dadurch realisieren, dass er nach Wahl des Benutzers diesen fragt in welcher
Reihenfolge die Aktionen ausgef�hrt werden sollen, oder aber die Reihenfolge
per Zufallsgenerator festlegt.

Die Transition von $s_4$ und $s_5$ zu $s_8$ schlie�t die paralelle Verzweigung
wieder.  Solche Transitionenen k�nnen nur genommen werden, wenn \emph{alle}
Quell-Steps aktiv sind.  Folglich kann diese Transition nur genommen werden
kann, wenn ihr Guard zu \emph{true} evaluiert wird, und ferner die beiden
Steps $s_4$ und $s_5$ aktiv sein.


\subsection{Zust�nde}


Der globale Zustand eines Programmes ist gegeben durch die Variablenbelegungen
und die Menge der aktiven Steps.  




%%% Local Variables: 
%%% mode: latex
%%% TeX-master: "main"
%%% End: 
