


\section{Konventionen und Spielregeln}
\label{sec:konventionen}

Neben den gesondert ausgeteilten und diskutierten \emph{CVS-Spielregeln}
sollen folgenden Dinge beachtet werden.

\begin{itemize}
\item \textbf{Makefile}: jedes Paket, d.h.\ die Wurzel des entsprechenden
  Unterverzeichnisses soll ein \texttt{Makefile} enhalten. Als erstes Target
  mu� \texttt{make all} unterst"utzt werden, welche f"ur das Paket den Java
  Bytecode erzeugt. Daneben soll ein \texttt{make clean} unterst"uzt werden,
  welches den byte-code und tempor"are Dateien wieder entfernt. Ein einfaches
  Beispiel f"ur ein passendes Makefile findet sich im Unterverzeichnis 
  \texttt{src/templates/}.
\item \textbf{Dokumentation:} Der \Java-Code soll sinnvoll kommentiert und
  dokumentiert werden.  F�r Information �ber die Implementierung, die f�r die
  Mit-Entwickler von Interesse ist, soll dies mittels \javadoc{} passieren.
  Dies betrifft insbesondere die Methoden, die zur Schnittstelle mit anderen
  Paketen geh�ren. Zur �ffentlichen (aber projekt-interne) Dokumentation
  geh�rt auch der Name der Entwickler und die Version. Weitere sinnvolle
  Information kann den \emph{Status} der Methode, Klasse, oder sonstigen
  Programmteils betreffen. Beispielsweise, ob die Implementiereung noch
  virtuell ist (als stub), ob nur Teile der vereinbarten Funktionalit"at
  bereitstehen, ob die Funktionalit"at noch ungetested ist, ob Fehler bekannt
  sind eind.
  
  Die \javadoc-Kommentare dienen dazu, die 
\item 
\end{itemize}


%%% Local Variables: 
%%% mode: latex
%%% TeX-master: "main"
%%% End: 
