\section{Model-Checker}
\label{sec:assert}


\team{}

Es soll die M�glichkeit gegeben werden, die \emph{Korrektheit} des
eingegebenen SFC's zu �berpr�fen. Dies soll in einfacher Weise dadurch
geschehen, da� �berpr�ft wird, ob auf allen Ausf�hrungen des Programmes die
\emph{Assertions} nicht verletzt sind.



\medskip

Im wesentlichen wird eine \emph{Graphsuche} implementiert werde: die
abstrakte Syntax wird in einem ersten Schritt in einen (expliziten oder
impliziten) Graphen �bersetzt, der danach mittels \emph{Tiefensuche} nach
Verletzung der Zusicherungen abgesucht wird.


Die Semantik der Prozesse ist in Anhang~\ref{sec:semantik} informell
beschrieben.


\subsubsection*{Schnittstelle}

Im wesentlichen mit der Gui (Abschnitt~\ref{sec:gui}): 

Methode \texttt{start\_modcheck} mit Parameter eines Programmes in
abstrakter Syntax. R�ckgabe: \emph{noch zu kl�ren}: entweder mittels
Ausnahme oder boolescher Wert. Auf jeden Fall wird der Gui der Zustand,
zur�ckgegeben wo die Verletzung auftritt. Wie die Gui darauf reagiert,
ist nicht Sache des Modelcheckerpaketes. (z.B.\ k�nnte der Zustand
gehighlighted werden.)







%%% Local Variables: 
%%% mode: latex
%%% TeX-master: "main"
%%% End: 
