
\section{�bersetzung nach SMV}
\label{sec:smv}


\SMV{} \cite{smv:gettingstarted} \cite{smv:language} ist ein weit verbreiteter
symbolischer Model-Checker.  Model-Checker werden benutzt, um festzustellen,
ob endliche Systeme bestimmte Eigenschaften besitzen.  Man k�nnte z.B.  die
Airbag-Steuerung eines Autos pr�fen wollen, ob es beispielsweise m�glich ist,
dass der Beifahrerairbag trotz Deaktivierung aktiviert werden kann.

Diese �berpr�ften Eigenschaften werden �blicherweise in einer \emph{temporalen
  Logik} beschrieben, in dieser ist es m�glich Eigenschaften der Art
"`immer/zu jeder Zeit gilt $a \lor b$ "' oder auch "`irgendwann gilt $a \land
b \land c$"' zu spezifizieren.

\SMV{} besitzt eine eigene Eingabesprache, in der auf unterschiedliche Art und
Weise Systeme spezifiziert werden k�nnen.  Hier ein Beispiel, wie man eine
sogenannte \emph{State machine} modellieren kann:
\begin{verbatim}
MODULE main
VAR
  request : boolean;
  state : {ready,busy};
ASSIGN
  init(state) := ready;
  next(state) := case
                   state = ready & request : busy;
                   1 : {ready,busy};
                 esac;
\end{verbatim}
Es werden zun�chst zwei Variablen deklariert, eine boolesche \texttt{request}
sowie eine enumerative \texttt{state}.  Letztere kann die Werte \texttt{ready}
und \texttt{busy} annehmen.  Im \texttt{ASSIGN}-Part wird das Verhalten des
Systems spezifiziert.  Initial wird die Variable \texttt{state} mit
\texttt{ready} belegt.  Im \texttt{next(state)}-Teil wird beschrieben, wie
sich der Wert von \texttt{state} abh�ngig vom aktuellen Zustand ver�ndert.
Wie man sieht ist nichts �ber \texttt{request} ausgesagt; folglich kann diese
Variable in jedem Schritt einen beliebigen Wert annehmen.

Die Aufgabe wird nun darin bestehen, ein SFC, welches ausschlie�lich boolesche
Varibalen besitzt, in diese Eingabesprache zu �bersetzen.  Diese �bersetzung
soll nat�rlich die Semantik des SFC's erhalten.  Somit sollen alle
Ausf�hrungssequenzen und erreichbaren Zust�nde des SFC's auch in der
�bersetzung vorhanden bzw. erreichbar sein.  Danach k�nnen dann Eigenschaften
des erhaltenen SMV-Systems gepr�ft werden.  Da die �bersetzung die Semantik
erh�lt, gelten diese Eigenschaften dann auch f�r das urspr�ngliche SFC.

%\cite{smv:manual}


%%% Local Variables: 
%%% mode: latex
%%% TeX-master: "main"
%%% End: 
