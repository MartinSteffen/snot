\section{Simulator}
\label{sec:simulator}

\team{}



Interaktive Simulation eines Programmes ist deren schrittweise Ausf�hrung,
soda� der Benutzer die Schritte initiieren und sie anhand der
Quell-\Snot-Prozesse nachvollziehen kann.  Der Simulator realisiert die
\emph{Semantik} aus Anhang~\ref{sec:semantik}.

Die Funktionalit�t umfa�t folgende Punkte:

\paragraph{Berechnung des Nachfolgezustandes:}
Der Algorithmus zur Berechnung des Nachfolgezustandes soll implementiert
werden.

\paragraph{Anzeige eines Schrittes:}
Der vom Simulator genommene Schritt mu� im Editor angezeigt werden. Dazu
wird die Highlight-Funktion des Editors genutzt.


Weiteres f�r erweiterte Funktionalit�t, was in der ersten Stufe
unber�cksichtigt bleibt:
\begin{itemize}
\item Back-stepping
\item Aufzeichnen (und Speichern) der genommenen Schritte.
\end{itemize}




%%% Local Variables: 
%%% mode: latex
%%% TeX-master: "main"
%%% End: 
