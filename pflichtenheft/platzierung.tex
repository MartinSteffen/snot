
\section{Platzierung}
\label{sec:platziering}


\team{--}

Der Editor erlaubt es, SFC's frei-hand zu zeichnen. Daneben
soll es m�glich sein, die Koordinaten der Transitionssysteme automatisch
zu berechnen.  Dazu mu� ein \emph{Graphplazierungsalgorithmus} entworfen
und implementiert werden.  Die SFC's
sollen m�glichst ``sch�n'' dargestellt werden.



\subsubsection*{Schnittstelle}
Gui und Editor. Die Graphplatzierung darf von gecheckter Syntax ausgehen.
Was die Bedeutung der Koordinaten betrifft: siehe den entsprechenden
Abschnitt beim Editor (Abschnitt~\ref{sec:editor}).
%  Da der Editor genau ein
% Fenster pro Prozess bereitstellt, m�ssen nur diese vom Platzierungsgruppe
% positioniert werde, nicht ganze Programme. Gui �bernimmt die
% Benutzerf�hrung in eigener Regie (nur ein Prozess soll positioniert werden,
% z.B., derjenige, dessen Fenster den Fokus hat), oder alle Prozesse sollen
% positioniert werden.

Angebote: eine Methode \texttt{position\_sfc}, die ein SFC in abstrakter
Syntax nimmt und ihn mit Koordinaten zur�ckgibt. Ob dies ebenfalls ein
Objekt der abstrakten Syntax ist oder einer anderen Datenstruktur, wurde
noch nicht festgelegt (siehe die Diskussion in Zusammenhang mit dem Editor
in Abschnitt~\ref{sec:editor}).

F�r den Anfang sei davon ausgegangen, da� alle Steps \emph{gleich gro�}
seien und Kanten als Geraden dargestellt werden.



\paragraph{Erweiterungsm�glichkeiten:} In einem ersten Schritt sollen
\emph{die Steps} plaziert werden, und die Transitionen als
\emph{Geraden} dazwischen. Falls Zeit ist, kann man versuchen,
\emph{gebogene} Transitionen zeichnen (d.h. auch berechnen!) zu lassen.
Sonstige Erweiterungsm�glichkeiten: Steps verschiedener Gr��en,
Ber�cksichtigung der Gr��e der Labels etc.











%%% Local Variables: 
%%% mode: latex
%%% TeX-master: "main"
%%% End: 
