\section{Graphische Benutzerschnittstelle (Gui)}
\label{sec:gui}

\team{}

\Snot{} besteht aus verschiedenen Komponenten, die ihrerseits mit dem
Benutzer interagieren. Es gibt eine �bergeordnete Schnittstelle, die
folgende Aufgaben bew�ltigt:


\begin{itemize}
\item\textbf{Start:} Beim Start einer \Snot-Session erscheint ein Fenster,
  von dem aus es m�glich ist, verschiedene Komponenten des Systems
  aufzurufen.
\item\textbf{Abh�ngigkeitsverwaltung:} Eine Simulation kann erst dann
  aufgerufen werden, wenn das Programm syntaktisch korrekt ist.  Das
  gleiche gilt f�r den Modelchecker. Die Aufgabe besteht darin, eine
  Definition der Abh�ngigkeiten zwischen den Komponenten festzulegen und
  sie im Tool zu implementieren.
\item\textbf{Sessionsverwaltung:} (2te Priorit�t) Es soll m�glich sein,
  eine Session (ge�ffnete Fenster, geladene Dateien, gew�hlte Optionen) zu
  speichern.  Eine gespeicherte Session sollte wieder hergestellt werden
  k�nnen.
\end{itemize}
Die Benutzeroberfl�che \emph{integriert} alle anderen Komponenten, aus
diesem Grund ist in dieser Gruppe besonders auf die \emph{Konsistenz} bzw.\ 
Verletzung dieser zu achten. Falls wir eine eigene Test-Gruppe bekommen,
dann kann diese einenn Teil der Verantwortung f"ur die Konsistenz
"ubernehmen. Die Arbeit sollte vorzugsweise von einer Gruppe mit
8st"undigen Teilnehmern bearbeitet werden, bzw.\ nicht von 100\%-en
\Java/C++-Einsteigern. 

\subsubsection*{Schnittstellen}

Mit allen anderen Paketen. Siehe die entsprechenden Abschnitte dort.




%%% Local Variables: 
%%% mode: latex
%%% TeX-master: "main"
%%% End: 
