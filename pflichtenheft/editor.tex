\section{Editor}
\label{sec:editor}

\team{}


Es wird ein graphischer \emph{Editor} f�r die SFC's mit den folgenden
Eigenschaften implementiert:

\begin{itemize}
\item\textbf{Aufbau:} Es soll m�glich sein, ein SFC aus \emph{Schablonen}
  von \emph{Schritten} (\textit{steps}), Transitionen und Pfaden zu
  zeichnen. Ein Vorschlag, wie SFCs aussehen k�nnen, ist in
  Abbildung~\ref{fig:SFC} zu sehen.
\item \textbf{Speichern und Laden:}
  Die Systeme sollen gespeichert und geladen werden k�nnen.
\item \textbf{Selektieren:} Einzelne Komponenten sollen selektiert werden
  k�nnen. Das dient zur Vorbereitung weiterer Aktionen.
\item \textbf{L�schen \& Kopieren:} Es soll m�glich sein, selektierte
  Komponenten zu entfernen und zu kopieren.
\item \textbf{Highlight}: der Editor soll eine Highlightfunktion zur
  Verf�gung stellen.  Es soll m�glich sein, bestimmte Schritte und
  Transitionen  hervorzuheben.
\end{itemize}







\subsubsection*{Schnittstelle}

Mit der Gui (Abschnitt~\ref{sec:gui}). Die Aufgabenverteilung zwischen Gui
und Editor ist zu diskutieren. Desweiteren mit dem Simulator
(Abschnitt~\ref{sec:simulator}). 

Auf jeden Fall: eine Methode \texttt{highlight\_state}, als �bergabe
entweder
\begin{itemize}
\item der Bezeichner des Zustandes, oder
\item der Zustand als Objekt.
\end{itemize}
Die Wahl mu� mit dem Simulator oder der Gui vereinbart werden, abh�ngig
davon, wer die Methode aufruft.

Eine \emph{wichtige} Schnittstelle (wie bei allen) ist die abstrakte Syntax.
Um das Zeichnen zu unterst�tzen, m�ssen eventuell \emph{Koordinaten} in die
abstrakte Syntax mit aufgenommen werden, dies ist zu diskutieren.

Die Aufgabe sollte vorzugsweise von einem 8-st�ndigen Team �bernommen
werden.


%%% Local Variables: 
%%% mode: latex
%%% TeX-master: "main"
%%% End: 
