
\section{Parser}
\label{sec:parser}


\team{Karsten Stahl, Martin Steffen}




Es soll eine nicht-graphische einfache Sprache als Eingabesprache erlaubt
sein.  Die Sprache soll in \Snot{} so unterst�tzt werden, da� man textuelle
Spezifikationen eingeben kann, ohne da� man auf die graphische Darstellung
verzichten mu�.  Die graphische Darstellung der Zust�nde wird von \Snot{}
berechnet.


Im ersten Schritt der Transformation (in diesem Modul) wird das textuelle
Programm geparst und als abstrakter Syntaxbaum (ohne graphische
Platzierung) dargestellt.


Die Implementierung wird \textsl{JLex} und \textsl{CUP} verwenden, welche
auf \texttt{\home{java}} installiert sind.

\subsubsection*{Schnittstelle}
Mit der Gui (Abschnitt~\ref{sec:gui}). Es wird eine Methode
\texttt{parse\_file} zur Verf�gung gestellt. Der Parameter ist ein String,
welcher die Datei bezeichnet, die das Programm enth�lt.  Die Dateien sollen
als Standard-Extension \texttt{.snot}-besitzen. Der Parser kann die
Ausnahme \texttt{Parser\_Exception} werfen. W�nschenswert ist, da"s der
Parser zumindest die Zeilennummer des Fehlers in der Ausnahme zur�ckgibt.

Eine weitere Schnittstelle ist vom \emph{Editor}
(Abschnitt~\ref{sec:editor}) gefordert: Das Parserpaket soll f�r den Editor
das \emph{parsen} eines \emph{Ausdruckes} (also einer \texttt{absynt.Expr})
bereitstellen. Die Eingabe soll ein \emph{String} sein. Bei Fehlschlag soll
eine Ausnahme geworfen werden.





%%% Local Variables: 
%%% mode: latex
%%% TeX-master: "main"
%%% End: 
