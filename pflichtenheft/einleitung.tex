\section{Einleitung}
\label{sec:einleitung}

Das Dokument beschreibt informell die Funktionalit�t von \Snot, einem
graphischen Analysewerkzeug f�r SFCs (\textit{\textbf{S}equential
  fu\textbf{n}ction charts m\textbf{o}deling \textbf{t}ool}).






Der Kern der Implementierung, um den sich alles zu gruppieren hat, ist die
\emph{abstrakte Syntax.}

%Sie ist (ohne graphische Komponenten) in
%Anhang~\ref{sec:AbstrakteSyntax} beschrieben.  

Die weiteren Abschnitte skizzieren Teilaufgaben des Projektes, die jeweils
als ein \emph{Paket} implementiert werden. Die \emph{optionalen} Aufgaben
sind sekund�r und werden nur hinzugenommen, falls es mehr Gruppen als
Aufgaben gibt oder falls eine Gruppe sehr schnell fertig ist.


Insbesondere wird das Dokument f�r jedes Paket
\begin{itemize}
\item die von ihr bereitgestellte Funktionalit�t, und
\item die von den anderen Gruppen erwartete Funktionalit�t festlegen.
\end{itemize}
Dies gilt vor allem f�r die Gruppe, die die \emph{Integration} �ber die
graphische Benutzerschnittstelle �bernimmt (Abschnitt~\ref{sec:gui}).

Da wir \emph{fr�h} mit der \emph{Integration} beginnen wollen, liegt die
Priorit�t hierbei auf fr�hzeitiger Bereitsstellung der versprochenen
Methoden, ohne da� dabei die Funktionalit�t bereits erbracht werden mu�
(als \textit{stubs}). Siehe hierzu auch den angegebenen Zeitplan.


Von unserer Seite wird eine Implementierung der abstrakten Syntax
(Abschnitt~\ref{sec:AbstrakteSyntax}) geliefert und ein globaler Rahmen
f�r das Projekt (Versionskontrolle etc.).

Falls man aus der Sicht seiner eigenen Gruppe �nderungs- oder
Erweiterungsw�nsche in Bezug auf die Klassen der abstrakten Syntax hat,
sollte man sie auch sobald wie m�glich anmelden, bzw.\ nach passender
Warnung an alle selber implementieren.


%%% Local Variables: 
%%% mode: latex
%%% ispell-dictionary: "deutsch"
%%% TeX-master: "main"
%%% End: 
