
\section*{Zeitplan}
\label{sec:zeitplan}


Die Dauer des Projektes ist festgelegt durch die Dauer des Sommersemesters
2001: Beginn ist der 23.\ April, Ende der 27.\ Juli. Das sind rechnerisch
\emph{13.\ Wochen.} Tabelle~\ref{tab:semesteruebersicht} stellt die geplanten
Termine und Ziele f"ur das Projekt zusammen. 

\begin{table}[htbp]
  \begin{center}
    \begin{tabular}{rrp{9cm}}
      Woche & Datum & Ziel 
      \\\hline
      1. & 23.\ April  &  erste Grobspezifikation; erste Besprechung;
      cvs-Repositorium fertig
      \\
      2. & 30.\ April & Modul- u.\  Gruppeneinteilung, technische
      Rahmenbedingungen klar (accounts, Zugriff, java etc.), Handouts
      durchgearbeitet, abstrakte Syntax funktionsf�hig
      \\
      3. &  7.\ Mai & klar(ere) Vorstellungen der gew�hlten Aufgabe,
      Gemeinschaftsbesprechung  der Schnittstellen, eventl.\ Vorstellungen
      "uber \Java-Spezifika der Library
      \\
      4.  & 14.\ Mai  & erster (nicht funktionsf�higer)
      Integrations/Kompilationstest (``stubs'')
      \\
      14. & 23.\ Juli & Abschlu�, Review-Treffen, Schlu"sdemo
    \end{tabular}
    \caption{Semester�bersicht}
    \label{tab:semesteruebersicht}
  \end{center}
\end{table}

Der wirklich feste Termin ist das Semesterende. Wichtiger noch als das
sklavische Einhalten der sich vorgenommenen Termine und Ziele ist
gegebenenfalls, rechtzeitig zu erkennen, da"s ein Ziel sich als
unrealistisch herausstellt und dies auch in den Besprechungen offen,
begr"undet und realistisch\footnote{Ein oft geh"ortes Symptom einer
  unrealistischen Lageeinsch"atzung geht ungef"ahr so: ``Wir sind die
  letzten drei Wochen nicht viel weiter gekommen, aber das ist kein
  Problem, denn von nun an werden wir dreimal so schnell arbeiten.''}  zur
Sprache zu bringen, so da"s man geeignet und rechtzeitig darauf reagieren
kann, z.B., durch Redefinition der Ziele, Umlagern der Last oder
"ahnlichem. Wichtiger und am Ende des Projektes befriedigender, als viele
Features ein bischen und wackelig realisiert zu haben ist es, eine
sinnvolle, geringere Auswahl vollst"andig und sicher implementiert zu
haben.








%%% Local Variables: 
%%% mode: latex
%%% TeX-master: "main"
%%% End: 
