
\section*{Zeitplan}
\label{sec:zeitplan}


Die Dauer des Projektes ist festgelegt durch die Dauer des Sommersemesters
2001: Beginn ist der 23.\ April, Ende der 27.\ Juli. Das sind rechnerisch
\emph{13.\ Wochen.} Tabelle~\ref{tab:semesteruebersicht} stellt die geplanten
Termine und Ziele f�r das Projekt zusammen. 

\begin{table}[htbp]
  \begin{center}
    \begin{tabular}{rrp{9cm}}
      Woche & Datum & Ziel 
      \\\hline
      1. & 23.\ April  &  erste Grobspezifikation; erste Besprechung;
      cvs-Repositorium fertig
      \\
      2. & 30.\ April & Modul- u.\  Gruppeneinteilung, technische
      Rahmenbedingungen klar (accounts, Zugriff, java etc.), Handouts
      durchgearbeitet, abstrakte Syntax funktionsf�hig
      \\
      3. &  7.\ Mai & klar(ere) Vorstellungen der gew�hlten Aufgabe,
      Gemeinschaftsbesprechung  der Schnittstellen, eventl.\ Vorstellungen
      �ber \Java-Spezifika der Library
      \\
      4.  & 14.\ Mai  & erster (nicht funktionsf�higer)
      Integrations/Kompilationstest (``stubs'')
      \\
      5.  & 22.\ Mai  & 
      \textbf{konkrete Schnittstellen (Java!)} vorhanden,
      Integrations/Kompilationstest in dieser Woche
      \\
      14. & 23.\ Juli & Abschlu�, Review-Treffen, Schlu�demo
    \end{tabular}
    \caption{Semester�bersicht}
    \label{tab:semesteruebersicht}
  \end{center}
\end{table}

Der wirklich feste Termin ist das Semesterende. Wichtiger noch als das
sklavische Einhalten der sich vorgenommenen Termine und Ziele ist
gegebenenfalls, rechtzeitig zu erkennen, da� ein Ziel sich als
unrealistisch herausstellt und dies auch in den Besprechungen offen,
begr�ndet und realistisch\footnote{Ein oft geh�rtes Symptom einer
  unrealistischen Lageeinsch�tzung geht ungef�hr so: ``Wir sind die
  letzten drei Wochen nicht viel weiter gekommen, aber das ist kein
  Problem, denn von nun an werden wir dreimal so schnell arbeiten.''}  zur
Sprache zu bringen, so da� man geeignet und rechtzeitig darauf reagieren
kann, z.B., durch Redefinition der Ziele, Umlagern der Last oder
�hnlichem. Wichtiger und am Ende des Projektes befriedigender, als viele
Features ein bischen und wackelig realisiert zu haben ist es, eine
sinnvolle, geringere Auswahl vollst�ndig und sicher implementiert zu
haben.








%%% Local Variables: 
%%% mode: latex
%%% TeX-master: "main"
%%% End: 
