


\input{switches}

\documentclass[11pt,german]{article}
\usepackage{hevea}
%\usepackage{url}
\usepackage{babel}
\usepackage{epsfig}
\usepackage{tfheader}
\usepackage{a4wide}
\usepackage[latin1]{inputenc}

%\newcommand{\kommentar}[1]{[{\small\em #1}]

\newcommand{\Snot}{\textsc{Snot}}
\newcommand{\Java}{\textsc{Java}}
\newcommand{\javadoc}{\textsc{javadoc}}



\newcommand{\team}[1]{\textbf{Team:} #1\bigskip{}}



\newcommand{\bnfdef}{::=}
\newcommand{\bnfbar}{\ensuremath{\mid}}
\newcommand{\of}    {\ensuremath{\mathrel{:}}}

\newenvironment{diagram}{\begin{displaymath}}{\end{displaymath}}

%\newcommand{\inputcode}[2][Code]   {
%  {\small
%  \mbox{}
%  \newline
%  \mbox{}
%  \hrulefill
%  \verbatiminput{#1/#2.java}
%  \hrulefill}}




%\renewcommand{\inputcode}[2][Code]{
%  {\small
%  \mbox{}
%  \newline\nopagebreak{}
%  \mbox{}
%  \hrulefill
%  \lstinputlisting{#1/#2.java}
%  \hrulefill}}

\newenvironment{code}{%
%  \small\mbox{}\nopagebreak{}\mbox\hrulefill{}
  {\begin{lstlisting}{}}}{%
  {\end{lstlisting}}}


%\newenvironment{diagram}{\begin{displaymath}}{\end{displaymath}}











%%% Local Variables: 
%%% mode: latex
%%% TeX-master: t
%%% End: 



\title{{\huge\bf \textsl{S}equential Fu\textsl{n}ction Charts
    M\textsl{o}deling \textsl{T}ool (aka.\ \Snot)}}
%\author{}
%\url{}{Martin Steffen}}
\date{}



\ifweb

\htmlfoot{\hrulefill{}
  {\footnotesize Pages last (re-)generated \today}}
\renewcommand{\@bodyargs}{bgcolor="white" alink="red" vlink="\#407999"  link="\#7070ff"}  
\fi




\begin{document}
\vspace{-2cm}


%\pagestyle{empty}

\begin{rawhtml}
<a href="http://www.techfak.uni-kiel.de/">
  <img border=0 alt="[Technical Faculty]" height=20  src="/images/tflogo.gif"></a>
<hr>
\end{rawhtml}


\maketitle{}


\section*{Pflichtenheft}
\label{sec:pflichtenheft}

Die folgende Liste enth�lt das Pflichtenheft. Der erste Eintrag ist
jeweils der \emph{aktuelle.}


\begin{center}
  \begin{tabular}[t]{r@{\quad}l@{\quad\quad}l@{\quad\quad}p{9cm}}
    4.
    & 
    \url{pflichtenheft3/}{Version 4}
    &
    28.\ Mai 2001
    & 
    Listen der abstr.\ Syntax reimplementiert. Vereinbarungen
    von letzter Woche eingetragen
    \\
    3.
    & 
    \url{pflichtenheft3/}{Version 3}
    &
    17.\ Mai 2001
    & 
    kleinere �nderungen, Teams eingetragen
    \\
    2.
    & 
    \url{pflichtenheft2/}{Version 2}
    &
    5.\ Mai 2001
    & 
    kleinere �nderungen an der Abstrakten Syntax, SMV-Teil ein
    wenig ausf�hrlicher
    \\
    1.
    & 
    \url{pflichtenheft1/}{Version 1}
    &
    2.\ Mai 2001
    & 
    Originalversion zu Beginn
  \end{tabular}
\end{center}






\section{Zeitplan}
\label{sec:zeitplan}


%%% Local Variables: 
%%% mode: latex
%%% TeX-master: "main"
%%% End: 


\section*{Arbeitsgruppen}
\label{sec:gruppen}

Jedes der Teams bearbeitet i.d.R.\ ein Java-Paket. Die Beschreibung der
Aufgabenstellung findet sich im Pflichtenheft.

\begin{table}[htbp]
  \centering
  \begin{tabular}[t]{l@{\quad\quad}l}
     Paket  &  Gruppe 
     \\\hline
     Gui/Integration & 
     Hans Theman, Ingo Schiller
     \\
     Editor &
     Natalia Freudenberg, Andreas Lukosch
     \\
     Checks & Tobias Pugatschov, Dimitri Schultheis
     \\
     SMV-Backend
     &
     Tobias Kloss, Kevin Koeser
     \\
     Simulator
     &
     Carsten Heine, J�rn Fiebelkorn
  \end{tabular}
  \caption{Gruppeneinteilung}
  \label{tab:gruppen}
\end{table}



Die \emph{Mailadressen} der Teilnehmner finden sich (intern) unter
\begin{verbatim}
      $WORKDIR/Snot/org/gruppen.txt
\end{verbatim}





\section*{Baselines}
\label{sec:Baselines}


Die \emph{Baselines} sind hier zur schnelleren Referenz als
\emph{Javaarchiv} \texttt{snot\_v[x].jar} festgehalten, wobei \texttt{[x]}
die Nummber des Schnappschusses ist. Als Entwickler mit Zugriff auf das
Quellcoderepositorium kann man die Entwicklungsschritte sie nat�rlich
mittels \cvs{} wieder zur�ckholen. Zum Ausf�hren der Archivs speichere man
die Datei \texttt{snot\_v[x].jar} an einen geeigneten Platz, und setze

\begin{verbatim}
 export CLASSPATH=[geeigneter_platz]/snot_v[x].jar
\end{verbatim}





\section*{HTML-Dokumentation}
\label{sec:html-doc}



Zur schnelleren Orientierung wird die mit \javadoc{} dokumentierte
Implementierung im Netz bereitgestellt. Die Seiten werden bei gr��eren
Entwicklungsschritten aufgefrischt.

\begin{center}
  \importantxx{Dokumentation \url{doc}{Snot}}: momentan (\today): absynt,
  utils vorhanden, stubs von allen au�er editor
\end{center}









\section{Konventionen}
\label{sec:konventionen}

Neben den gesondert ausgeteilten \emph{CVS-Spielregeln} sollen folgenden
Dinge beachtet werden.

\begin{itemize}
\item \textbf{Makefile}: jedes Paket, d.h.\ die Wurzel des entsprechenden
  Unterverzeichnisses soll ein \texttt{Makefile} enhalten. Als erstes
  Target \texttt{make all} unterst"utzt werden, welche f"ur das Paket den
  Java Bytecode erzeugt. Daneben soll ein \texttt{make clean} unterst"uzt
  werden, welches den byte-code und tempor"are Dateien wieder entfernt. Ein
  sehr einfaches Beispiel f"ur ein passendes Makefile findet sich im Paket
  \texttt{absynt}. 
\item \texttt{Dokumentation:} Der 
\item 
\end{itemize}





%%% Local Variables: 
%%% mode: latex
%%% TeX-master: "main"
%%% End: 




\bibliographystyle{alpha}



\bibliography{string,etc,oop,crossref}


\end{document}



%%% Local Variables: 
%%% mode: latex
%%% TeX-master: t
%%% End: 

